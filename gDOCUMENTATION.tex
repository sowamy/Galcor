\documentclass{article}
%-----------------------------------------------------------------------------------------------------------------------------------------------------------------------------
% Packages
\usepackage[margin=1in,includefoot]{geometry}
\usepackage{enumitem}
%-----------------------------------------------------------------------------------------------------------------------------------------------------------------------------
\begin{document}
%-----------------------------------------------------------------------------------------------------------------------------------------------------------------------------
% Title Page
\title{GALCOR Documentation}
\author{Angelino Lefevers}
\maketitle
\thispagestyle{empty}
\cleardoublepage
%-----------------------------------------------------------------------------------------------------------------------------------------------------------------------------
% Table of Contents
\tableofcontents
\thispagestyle{empty}
\cleardoublepage
%-----------------------------------------------------------------------------------------------------------------------------------------------------------------------------
% Introduction
\setcounter{page}{1}
\section{Introduction}
GALCOR is an assistance program in which performs various calculations and functions for the user. GALCOR functionality can be accessed by either calling a variety of functions from a program script, or by using the GALCOR GUI.
\cleardoublepage
%-----------------------------------------------------------------------------------------------------------------------------------------------------------------------------
% gGENERATOR
\cleardoublepage
\section{gGENERATOR.c}

% -- Function Prototypes
\subsection{Function Prototypes}
{\renewcommand{\labelitemi}{$\triangleright$}
\begin{itemize}[noitemsep]
\footnotesize
\item int $^{\ast}$\textbf{gGENERATOR\_DCSignalGenerator0}(int \textit{size}, int \textit{amplitude}, int \textit{delay});
\item double $^{\ast}$\textbf{gGENERATOR\_DCSignalGenerator1}(int \textit{size}, double \textit{amplitude}, int \textit{delay});
\item int $^{\ast}$\textbf{gGENERATOR\_DCSignalInserter0}(int $^{\ast}$\textit{signalSource}, int \textit{index}, int \textit{sourceSize}, int \textit{amplitude}, int \textit{signalSize});
\item  double $^{\ast}$\textbf{gGENERATOR\_DCSignalInserter1}(int $^{\ast}$\textit{signalSource}, int \textit{index}, int \textit{sourceSize}, double \textit{amplitude}, int \textit{signalSize});
\item int $^{\ast}$\textbf{gGENERATOR\_ImpulseSignalGenerator0}(int \textit{size}, int \textit{amplitude}, int \textit{delay});
\item double $^{\ast}$\textbf{gGENERATOR\_ImpulseSignalGenerator1}(int \textit{size}, double \textit{amplitude}, int \textit{delay});
\item int $^{\ast}$\textbf{gGENERATOR\_RandomIntegerSignalGenerator}(int \textit{size}, int \textit{minRange}, int \textit{maxRange});
\normalsize
\end{itemize}
\cleardoublepage
% -- Functions
\subsection{Functions}

% -- --  DCSignalGenerator0
\subsubsection{gGENERATOR\_DCSignalGenerator0}
{\renewcommand{\labelitemi}{$\bullet$}
\begin{itemize}
\item Prototype:
	\begin{itemize}
	\item int $^{\ast}$\textbf{gGENERATOR\_DCSignalGenerator0}(int \textit{\textbf{size}}, int \textit{\textbf{amplitude}}, int \textit{\textbf{delay}});
	\end{itemize}
\item Description:
	\begin{itemize}[noitemsep]
	\item Consists of integer values
	\item Generates a constant DC signal of length \textit{size}
	\item Returns a pointer to the loaction of the generated DC signal
	\end{itemize}
\item Arguments:
	\begin{itemize}[noitemsep]
	\item \textit{int} \textit{\textbf{size}} := Number of samples in the signal to be generated.
	\item \textit{int} \textit{\textbf{amplitude}} := The amplitude of the DC signal to be generated.
	\item \textit{int} \textit{\textbf{delay}} := Number of samples the generated DC signal will be delayed from TIME 0.
	\end{itemize} 
\item Intermediate Variables:
	\begin{itemize}[noitemsep]
	\item \textit{int$^{\ast}$} \textbf{dcSignalSource} := Pointer to first element of the generated signal.
	\item \textit{int$^{\ast}$} \textbf{tSignal} := Temporary offset pointer used to iterate through and initialize the signal.
	\item \textit{int} \textbf{i, j} := Counters to be used in looping.
	\end{itemize}
\item Return:
	\begin{itemize}
	\item \textit{int$^{\ast}$} \textbf{dcSignalSource}
	\end{itemize}
\item Tested:
	\begin{itemize}
	\item \textbf{YES} - LIMITED
	\end{itemize}
\end{itemize}

% -- --  DCSignalGenerator1
\cleardoublepage
\subsubsection{gGENERATOR\_DCSignalGenerator1}
{\renewcommand{\labelitemi}{$\bullet$}
\begin{itemize}
\item Prototype:
	\begin{itemize}
	\item double $^{\ast}$\textbf{gGENERATOR\_DCSignalGenerator1}(int \textit{\textbf{size}}, double \textit{\textbf{amplitude}}, int \textit{\textbf{delay}});
	\end{itemize}
\item Description:
	\begin{itemize}[noitemsep]
	\item Consists of double floating-point values
	\item Generates a constant DC signal of length \textit{size}
	\item Returns a pointer to the loaction of the generated DC signal
	\end{itemize}
\item Arguments:
	\begin{itemize}[noitemsep]
	\item \textit{int} \textit{\textbf{size}} := Number of samples in the signal to be generated.
	\item \textit{double} \textit{\textbf{amplitude}} := The amplitude of the DC signal to be generated.
	\item \textit{int} \textit{\textbf{delay}} := Number of samples the generated DC signal will be delayed from TIME 0.
	\end{itemize} 
\item Intermediate Variables:
	\begin{itemize}[noitemsep]
	\item \textit{double$^{\ast}$} \textbf{dcSignalSource} := Pointer to first element of the generated signal.
	\item \textit{double$^{\ast}$} \textbf{tSignal} := Temporary offset pointer used to iterate through and initialize the signal.
	\item \textit{int} \textbf{i, j} := Counters to be used in looping.
	\end{itemize}
\item Return:
	\begin{itemize}
	\item \textit{double$^{\ast}$} \textbf{dcSignalSource}
	\end{itemize}
\item Tested:
	\begin{itemize}
	\item \textbf{YES} - LIMITED
	\end{itemize}
\end{itemize}

% -- -- DCSignalInserter0
\cleardoublepage
\subsubsection{gGENERATOR\_DCSignalInserter0}
{\renewcommand{\labelitemi}{$\bullet$}
\begin{itemize}
\item Prototype:
	\begin{itemize}
	\item int $^{\ast}$\textbf{gGENERATOR\_DCSignalGenerator0}(int $^{\ast}$\textit{\textbf{signalSource}}, int \textit{\textbf{index}}, int \textit{\textbf{sourceSize}}, int \textit{\textbf{amplitude}}, int \textit{\textbf{sourceSize}});
	\end{itemize}
\item Description:
	\begin{itemize}[noitemsep]
	\item Consists of integers
	\item Takes an array of integer elements of size \textit{sourceSize} and pointed to by \textit{signalSource}.
	\item Inserts a DC signal into the index value \textit{index} from the signal source.
	\item The DC signal inserted is \textit{signalSize} elements long and has an amplitude of \textit{amplitude}.
	\end{itemize}
\item Arguments:
	\begin{itemize}[noitemsep]
	\item \textit{int$^{\ast}$\textbf{signalSource}} := Pointer to the first element of the inputted signal.
	\item \textit{int \textbf{index}} :=Index value of where the generated DC signal is inserted.
	\item \textit{int \textbf{sourceSize}} := The number of elements in the inputted signal.
	\end{itemize} 
\item Intermediate Variables:
	\begin{itemize}[noitemsep]
	\item \textit{int$^{\ast}$} \textbf{dcSignalSource} := Pointer to first element of the generated signal.
	\item \textit{int$^{\ast}$} \textbf{tSignal} := Temporary offset pointer used to iterate through and initialize the signal.
	\item \textit{int} \textbf{i, j} := Counters to be used in looping.
	\end{itemize}
\item Return:
	\begin{itemize}
	\item \textit{int$^{\ast}$} \textbf{dcSignalSource}
	\end{itemize}
\item Tested:
	\begin{itemize}
	\item \textbf{YES} - LIMITED
	\end{itemize}
\end{itemize}
%-----------------------------------------------------------------------------------------------------------------------------------------------------------------------------
% gECON.c
\cleardoublepage
\section{gECON.c}

% -- Function Prototypes
\subsection{Function Prototypes}
{\renewcommand{\labelitemi}{$\triangleright$}
\begin{itemize}[noitemsep]
\footnotesize
\item double $^{\ast}$\textbf{gECON\_SellingPricePerUnit}(double \textit{constant\_a}, double \textit{constant\_b}, double \textit{demand});
\item double $^{\ast}$\textbf{gECON\_Demand}(double \textit{constant\_a}, double \textit{constant\_b}, double \textit{pricePerUnit});
\item double $^{\ast}$\textbf{gECON\_TotalRevenue0}(double \textit{price}, double \textit{demand});
\item double $^{\ast}$\textbf{gECON\_TotalRevenue1}(double \textit{constant\_a}, double \textit{constant\_b}, double \textit{demand});
\item double $^{\ast}$\textbf{gECON\_TotalCost}(double \textit{fixedCosts}, double \textit{variableCostsPerUnit}, double \textit{demand});
\item double $^{\ast}$\textbf{gECON\_Profit0}(double \textit{totalRevenue}, double \textit{totalCosts});
\item double $^{\ast}$\textbf{gECON\_Profit1}(double \textit{constant\_a}, double \textit{constant\_b}, double \textit{demand}, double \textit{fixedCosts}, double \textit{variableCostsPerUnit});
\normalsize
\end{itemize}
\cleardoublepage
{\renewcommand{\labelitemi}{$\bullet$}

% -- Functions
\subsection{Functions}

% -- -- SellingPricePerUnit
\subsubsection{gECON\_SellingPricePerUnit}
\begin{itemize}
\item Prototype:
	\begin{itemize}
	\item double \textbf{gECON\_SellingPricePerUnit}(double \textbf{\textit{constant\_a}}, double \textbf{\textit{constant\_b}}, double \textbf{\textit{demand}});
	\end{itemize}
\item Description:
	\begin{itemize}
	\item Returns the selling price per unit given constants \textit{constant\_a} and \textit{constant\_b}, and the demand \textit{demand}.
	\end{itemize}
\item Equation:     $p = a - bD$
	\begin{itemize}
	\item p := Price per Unit
	\item D := "Demand" - Total number of units sold.
	\item a, b := constants
	\item Conditions:
		\begin{itemize}
		\item $0 \le D \le \frac{a}{b}$
		\item $a > 0$
		\item $b > 0$
		\end{itemize}
	\end{itemize}
\item Arguments:
	\begin{itemize}[noitemsep]
	\item \textit{double \textbf{constant\_a}} := Constant
	\item \textit{double \textbf{constant\_b}} := Constant
	\item \textit{double \textbf{demand}} := The total number of units sold.
	\end{itemize} 
\item Intermediate Variables: \textbf{NONE}
\item Return:
	\begin{itemize}
	\item \textit{double} \textbf{pricePerUnit}
	\end{itemize}
\item Tested:
	\begin{itemize}
	\item \textbf{YES} - LIMITED
	\end{itemize}
\end{itemize}

% -- -- Demand
\cleardoublepage
\subsubsection{gECON\_Demand}

\begin{itemize}
\item Prototype:
	\begin{itemize}
	\item double \textbf{gECON\_Demand}(double \textbf{\textit{constant\_a}}, double \textbf{\textit{constant\_b}}, double \textbf{\textit{pricePerUnit}});
	\end{itemize}
\item Description:
	\begin{itemize}
	\item Returns the selling price per unit given constants \textit{constant\_a} and \textit{constant\_b}, and the price per unit \textit{pricePerUnit}.
	\end{itemize}
\item Equation:     $D = \frac{a - p}{b}$
	\begin{itemize}
	\item D := "Demand" - Total number of units sold.
	\item p := Price per Unit
	\item a, b := constants
	\item Conditions:
		\begin{itemize}
		\item $b \ne 0$
		\end{itemize}
	\end{itemize}
\item Arguments:
	\begin{itemize}[noitemsep]
	\item \textit{double \textbf{constant\_a}} := Constant
	\item \textit{double \textbf{constant\_b}} := Constant
	\item \textit{double \textbf{pricePerUnit}} := The price per unit which was manufactured.
	\end{itemize} 
\item Intermediate Variables: \textbf{NONE}
\item Return:
	\begin{itemize}
	\item \textit{double} \textbf{demand}
	\end{itemize}
\item Tested:
	\begin{itemize}
	\item \textbf{YES} - LIMITED
	\end{itemize}
\end{itemize}

% -- -- Total Revenue 0
\cleardoublepage
\subsubsection{gECON\_TotalRevenue0}

\begin{itemize}
\item Prototype:
	\begin{itemize}
	\item double \textbf{gECON\_TotalRevenue0}(double \textbf{\textit{pricePerUnit}}, double \textbf{\textit{demand}});
	\end{itemize}
\item Description:
	\begin{itemize}
	\item Returns the total revenue given the \textit{pricePerUnit} and the \textit{demand} of the product.
	\end{itemize}
\item Equation:     $TR = pD$
	\begin{itemize}[noitemsep]
	\item TR := Total revenue earned from a production.
	\item p := Price per Unit
	\item D := Demand, or the total number of units sold after production.
	\end{itemize}
\item Arguments:
	\begin{itemize}[noitemsep]
	\item \textit{double \textbf{pricePerUnit}} := Price per Unit
	\item \textit{double \textbf{demand}} := Total number of units sold.
	\end{itemize} 
\item Intermediate Variables: \textbf{NONE}
\item Return:
	\begin{itemize}
	\item \textit{double} \textbf{totalRevenue}
	\end{itemize}
\item Tested:
	\begin{itemize}
	\item \textbf{YES} - LIMITED
	\end{itemize}
\end{itemize}

% -- -- Total Revenue 1
\cleardoublepage
\subsubsection{gECON\_TotalRevenue1}

\begin{itemize}
\item Prototype:
	\begin{itemize}
	\item double \textbf{gECON\_TotalRevenue1}(double \textbf{\textit{constant\_a}}, double \textbf{\textit{constant\_b}}, double \textbf{\textit{demand}});
	\end{itemize}
\item Description:
	\begin{itemize}
	\item Returns the total revenue given the constants \textit{constant\_a} and \textit{constant\_b},  and the \textit{demand} of the product.
	\end{itemize}
\item Equation:     $TR = (a - bD)D$
	\begin{itemize}[noitemsep]
	\item TR := Total revenue earned from a production.
	\item D := Demand, or the total number of units sold after production.
	\item a, b := Constants
	\item Conditions:
		\begin{itemize}
		\item $0 \le D \le \frac{a}{b}$
		\item $a > 0$
		\item $b > 0$
		\end{itemize}
	\end{itemize}
\item Arguments:
	\begin{itemize}[noitemsep]
	\item \textit{double \textbf{constant\_a}} := Constant
	\item \textit{double \textbf{constant\_b}} := Constant
	\item \textit{double \textbf{demand}} := Total number of units sold after production.
	\end{itemize} 
\item Intermediate Variables: \textbf{NONE}
\item Return:
	\begin{itemize}
	\item \textit{double} \textbf{totalRevenue}
	\end{itemize}
\item Tested:
	\begin{itemize}
	\item \textbf{YES} - LIMITED
	\end{itemize}
\end{itemize}

% -- -- Total Cost 0
\cleardoublepage
\subsubsection{gECON\_TotalCost0}

\begin{itemize}
\item Prototype:
	\begin{itemize}
	\item double \textbf{gECON\_TotalCost0}(double \textbf{\textit{fixedCosts}}, double \textbf{\textit{variableCosts}});
	\end{itemize}
\item Description:
	\begin{itemize}
	\item Returns the total costs of production given the \textit{fixedCosts} the \textit{variableCosts} of the production.
	\end{itemize}
\item Equation:     $C_T = C_F + C_V$
	\begin{itemize}[noitemsep]
	\item $C_T$ := Total costs in production.
	\item $C_F$ := Total fixed (initial) costs of production.
	\item $C_V$ := Total variable costs of production.
	\end{itemize}
\item Arguments:
	\begin{itemize}[noitemsep]
	\item \textit{double \textbf{fixedCosts}} := Fixed costs of the production.
	\item \textit{double \textbf{variableCosts}} := Variable costs of the production
	\end{itemize} 
\item Intermediate Variables: \textbf{NONE}
\item Return:
	\begin{itemize}
	\item \textit{double} \textbf{totalCosts}
	\end{itemize}
\item Tested:
	\begin{itemize}
	\item \textbf{YES} - LIMITED
	\end{itemize}
\end{itemize}

% -- -- Total Cost 1
\cleardoublepage
\subsubsection{gECON\_TotalCost1}

\begin{itemize}
\item Prototype:
	\begin{itemize}
	\item double \textbf{gECON\_TotalCost1}(double \textbf{\textit{fixedCosts}}, double \textbf{\textit{variableCostsPerUnit}}, double \textbf{\textit{demand}});
	\end{itemize}
\item Description:
	\begin{itemize}
	\item Returns the total costs of production given the \textit{fixedCosts}, the \textit{variableCostsPerUnit}, and the \textit{demand} of the production.
	\end{itemize}
\item Equation:     $C_T = C_F + ( c_v * D )$
	\begin{itemize}[noitemsep]
	\item $C_T$ := Total costs in production.
	\item $C_F$ := Total fixed (initial) costs of production.
	\item $c_v$ := Variable costs of production per unit.
	\item D := Total number of units sold after production.
	\end{itemize}
\item Arguments:
	\begin{itemize}[noitemsep]
	\item \textit{double \textbf{fixedCosts}} := Fixed costs of the production.
	\item \textit{double \textbf{variableCostsPerUnit}} := Variable costs per unit.
	\item \textit{double \textbf{demand}} := Total number of units sold after production.
	\end{itemize} 
\item Intermediate Variables: \textbf{NONE}
\item Return:
	\begin{itemize}
	\item \textit{double} \textbf{totalCosts}
	\end{itemize}
\item Tested:
	\begin{itemize}
	\item \textbf{YES} - LIMITED
	\end{itemize}
\end{itemize}

% -- -- Profit
\cleardoublepage
\subsubsection{gECON\_Profit}

\begin{itemize}
\item Prototype:
	\begin{itemize}
	\item double \textbf{gECON\_Profit}(double \textbf{\textit{totalRevenue}}, double \textbf{\textit{totalCosts}});
	\end{itemize}
\item Description:
	\begin{itemize}
	\item Returns the total costs of production given the \textit{fixedCosts}, the \textit{variableCostsPerUnit}, and the \textit{demand} of the production.
	\end{itemize}
\item Equation:     $P = TR - C_T$
	\begin{itemize}[noitemsep]
	\item P := Total profit made from production.
	\item TR := Total revenue earned from production.
	\item $C_T$ := Total costs in production.
	\end{itemize}
\item Arguments:
	\begin{itemize}[noitemsep]
	\item \textit{double \textbf{fixedCosts}} := Fixed costs of the production.
	\item \textit{double \textbf{variableCostsPerUnit}} := Variable costs per unit.
	\item \textit{double \textbf{demand}} := Total number of units sold after production.
	\end{itemize} 
\item Intermediate Variables: \textbf{NONE}
\item Return:
	\begin{itemize}
	\item \textit{double} \textbf{profit}
	\end{itemize}
\item Tested:
	\begin{itemize}
	\item \textbf{YES} - LIMITED
	\end{itemize}
\end{itemize}

% -- -- Revenue Maximized Demand
\cleardoublepage
\subsubsection{gECON\_RevenueMaximizedDemand}

\begin{itemize}
\item Prototype:
	\begin{itemize}
	\item double \textbf{gECON\_RevenueMaximizedDemand}(double \textbf{\textit{constant\_a}}, double \textbf{\textit{constant\_b}}, double \textbf{\textit{demand}});
	\end{itemize}
\item Description:
	\begin{itemize}
	\item Returns the number of units sold which would maximize the revenue earned from the production.
	\end{itemize}
\item Equation:     $\hat{D} = \frac{a}{2b}$
	\begin{itemize}[noitemsep]
	\item $\hat{D}$ := Total units sold to maximize the total revenue earned from production.
	\item a, b := Constants
	\item Conditions:
		\begin{itemize}
		\item $\frac{dTR}{dD} = a - 2bD = 0$
		\end{itemize}
	\end{itemize}
\item Arguments:
	\begin{itemize}[noitemsep]
	\item \textit{double \textbf{constant\_a}} := Constant
	\item \textit{double \textbf{constant\_b}} := Constant
	\item \textit{double \textbf{demand}} := Total number of units sold after production.
	\end{itemize} 
\item Intermediate Variables: \textbf{NONE}
\item Return:
	\begin{itemize}
	\item \textit{double} \textbf{revenueMaximizedDemand}
	\end{itemize}
\item Tested:
	\begin{itemize}
	\item \textbf{YES} - LIMITED
	\end{itemize}
\end{itemize}

% -- -- Revenue Maximized Demand
\cleardoublepage
\subsubsection{gECON\_BreakEvenPoints}

\begin{itemize}
\item Prototype:
	\begin{itemize}
	\item double \textbf{gECON\_BreakEvenPoints}(double \textbf{\textit{constant\_a}}, double \textbf{\textit{constant\_b}}, double \textbf{\textit{variableCostsPerUnit}}, double \textbf{\textit{fixedCosts}});
	\end{itemize}
\item Description:
	\begin{itemize}
	\item Returns the number of units which need to be sold after production for the total revenue earned from the production will equal to the total costs of the production. 
	\end{itemize}
\item Equation:     $BEP = \frac{-(a-c_v)\pm\sqrt{(a-c_v)^2-4(b)(C_F)}}{2(-b)} = \frac{i0\pm i1}{i2}$
	\begin{itemize}[noitemsep]
	\item BEP := Number of units which need to be sold for the total revenue to equal total costs.
	\item $C_F$ := The fixed or inital costs of the production
	\item $c_v$ := The costs per unit produced.
	\item a, b := Constants
	\end{itemize}
\item Arguments:
	\begin{itemize}[noitemsep]
	\item \textit{double \textbf{constant\_a}} := Constant
	\item \textit{double \textbf{constant\_b}} := Constant
	\item \textit{double \textbf{variableCostsPerUnit}} := The costs per each unit produced.
	\item \textit{double \textbf{fixedCosts}} := The fixed of initial costs of the production.
	\end{itemize} 
\item Structure and Type Definition:
	\begin{itemize}https://github.com/sowamy/Galcor
	\item 
\begin{verbatim}
struct breakEvenPoints {
    double low;
    double high;
};

typedef struct breakEvenPoints BEP;
\end{verbatim}
	\end{itemize}
\item Intermediate Variables: 
	\begin{itemize}
	\item \textit{double} \textbf{i0} := Intermediate variable 0.
	\item \textit{double} \textbf{i1} := Intermediate variable 1.
	\item \textit{double} \textbf{i2} := Intermediate variable 2.
	\item \textit{BEP} \textbf{out} := breakEvenPoints structure whose elements represent the low and high break even points.
	\end{itemize}
\item Return:
	\begin{itemize}
	\item \textit{double} \textbf{out}
	\end{itemize}
\item Tested:
	\begin{itemize}
	\item \textbf{YES} - LIMITED
	\end{itemize}
\end{itemize}
\cleardoublepage

\end{document}
